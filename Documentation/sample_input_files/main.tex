\documentclass{article}
\usepackage[utf8]{inputenc}
\usepackage{amsmath}

\title{Very Basic Environments}
\author{Connor Barlow}
\date{October 2020}

\newtheorem{thm}{Theorem}
\newtheorem{corr}{Corollary}

\begin{document}
\maketitle

\section{Lists}
\begin{enumerate}
    \item This is a numbered list.
    \item The numbers should be read aloud.
\end{enumerate}
\begin{itemize}
    \item This is a bullet list.
    \item The items have no intrinsic ordering.
\end{itemize}
\begin{description}
    \item[Bob] What type of list is this?
    \item[Bill] Its for listing items that have a corresponding name!
    \item[Bob] Neat.
\end{description}
\section{Simple Table}
\par
Here's a very ugly table to show different combinations of lines.
\par
\begin{tabular}{|l||cr}
1 & 2 & 3 \\
4 & 5 & 6 \\
\hline
7 & 8 & 9 \\
\hline \hline
\end{tabular}
\section*{This Section Isn't Numbered}
\begin{verbatim}
\textbf{This} text is not formatted in \textit{any} way,
\end{verbatim}
but \textit{this} \textbf{is}! Now here's some math related things.\\
Here's some varieties of math mode. First some line separated math modes \dots
\[f(x)=x\]
$$g(x)=\frac{1}{f(x)}$$
\dots and now inline math modes, \($lim$_{x\rightarrow\infty}f(x)=\infty\) and $\int g(x)=\ln(x)$. For something a bit more sophisticated we have equations and theorems.
\begin{equation}
h(x)=x^2
\end{equation}
\begin{eqnarray}
\text{short} &= -\frac{b+\sqrt{b^2-4ac}}{2a}\\
\text{looooooongVariableName} &= -\frac{b-\sqrt{b^2-4ac}}{2a}
\end{eqnarray}
\begin{thm}
This is my theorem.
\end{thm}
\begin{corr}
This is my corollary.
\end{corr}
\begin{center}
And now a couple basic formatting environments.
\end{center}
\begin{flushright}
These shouldn't change too much besides documents structure.
\end{flushright}
\begin{flushleft}
And their typical use won't be as contrived as this.
\end{flushleft}
\end{document}
